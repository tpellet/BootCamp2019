\documentclass[11pt]{article}

% Extract
% \usepackage[active,
%             generate=topology_definitions,
%             %extract-cmd={section},
%             extract-env={definition,algorithm}]{extract}

% \usepackage[active,
%             generate=topology_theorems,
%             %extract-cmd={section},
%             extract-env={theorem,corollary,claim}]{extract}

% \begin{extract*}

%%%%%%%%%%%%%%%%%%%%%%%%%%%%%%%%%%%%%%%%%%%%%%%%%%%%%%%%%%%%%%%%%%%%%%%%%%%%%%%%%%

% Packages

% AMS 
\usepackage{amsmath, amssymb, amsthm, amsbsy}
% Geometry
\usepackage{geometry}
% Colors
\usepackage[usenames,dvipsnames]{xcolor}
% Figures
\usepackage{graphicx}
\usepackage{float}
% Multi column lists
\usepackage{multicol}
% Subfigures
\usepackage{caption}
\usepackage{subcaption}
% Caligraphic
\usepackage{mathrsfs}
\usepackage{bbm}
% Bold
\usepackage{bm}
% algos
\usepackage[linesnumbered, lined, ruled]{algorithm2e}
% Spacing 
\usepackage{setspace}
% Refs/links
\usepackage[colorlinks=true, citecolor=Blue, linkcolor=blue]{hyperref}
\newcommand\myshade{85}
\colorlet{mylinkcolor}{violet}
\colorlet{mycitecolor}{PineGreen}
\colorlet{myurlcolor}{Aquamarine}

\hypersetup{
  linkcolor  = mylinkcolor!\myshade!black,
  citecolor  = mycitecolor!\myshade!black,
  urlcolor   = myurlcolor!\myshade!black,
  colorlinks = true,
}
% Bibliography
\usepackage{filecontents}
\usepackage{natbib}
% Indent
\usepackage{indentfirst}
% Pretty lists
\usepackage{enumitem}
\setlist[enumerate]{itemsep=2pt,topsep=3pt}
\setlist[itemize]{itemsep=2pt,topsep=3pt}
\setlist[enumerate,1]{label=(\roman*)}

% Code
\usepackage{listings}

% Appendix
\usepackage[toc,page]{appendix}

% Math
\usepackage{mathtools}
\usepackage{xparse}

% Equation numbering
\numberwithin{equation}{section}

% Use more than one optional parameter in a new commands
\usepackage{xargs}                      
% Todo
\usepackage[colorinlistoftodos,prependcaption,textsize=normalsize]{todonotes}
\newcommandx{\unsure}[2][1=]{\todo[linecolor=red,backgroundcolor=red!25,bordercolor=red,#1]{#2}}
\newcommandx{\change}[2][1=]{\todo[linecolor=blue,backgroundcolor=blue!25,bordercolor=blue,#1]{#2}}
\newcommandx{\info}[2][1=]{\todo[linecolor=OliveGreen,backgroundcolor=OliveGreen!25,bordercolor=OliveGreen,#1]{#2}}
\newcommandx{\improvement}[2][1=]{\todo[linecolor=Plum,backgroundcolor=Plum!25,bordercolor=Plum,#1]{#2}}
\newcommandx{\thiswillnotshow}[2][1=]{\todo[disable,#1]{#2}}

%%%%%%%%%%%%%%%%%%%%%%%%%%%%%%%%%%%%%%%%%%%%%%%%%%%%%%%%%%%%%%%%%%%%%%%%%%%%%%%%%%

% Document Settings

% Figure path
\graphicspath{{./figures/}}
% Matrix columns
\setcounter{MaxMatrixCols}{10}
% So pages will break inside long equation environments
\allowdisplaybreaks
% Font
\usepackage{mathpazo} 
\linespread{1.05}  
%\usepackage{courier}
% Geometry
\geometry{left=1in,right=1in,top=1in,bottom=1in}
% Counters
\setcounter{tocdepth}{2}
\setcounter{secnumdepth}{3}

%%%%%%%%%%%%%%%%%%%%%%%%%%%%%%%%%%%%%%%%%%%%%%%%%%%%%%%%%%%%%%%%%%%%%%%%%%%%%%%%%%

% Colors

\definecolor{Tm}{rgb}{0,0,0.80}
\newcommand{\defi}[1]{\textcolor{MidnightBlue}{\bf #1}}

%%%%%%%%%%%%%%%%%%%%%%%%%%%%%%%%%%%%%%%%%%%%%%%%%%%%%%%%%%%%%%%%%%%%%%%%%%%%%%%%%%

% Environments

\theoremstyle{plain}
\newtheorem{theorem}{\color{ForestGreen}{\textbf{Theorem}}}[section]
\newtheorem{claim}{\color{ForestGreen}{\textbf{Claim}}}[section]
\newtheorem{lemma}[theorem]{\color{ForestGreen}{\textbf{Lemma}}}
\newtheorem{proposition}[theorem]{\color{ForestGreen}{\textbf{Proposition}}}
\newtheorem{corollary}[theorem]{\color{ForestGreen}{\textbf{Corollary}}}
\newtheorem{axiom}[theorem]{\color{ForestGreen}{\textbf{Axiom}}}
\newtheorem{conjecture}[theorem]{Conjecture}
\newtheorem{case}[theorem]{Case}
\newtheorem{conclusion}[theorem]{Conclusion}
\newtheorem{criterion}[theorem]{Criterion}
\newtheorem{notation}[theorem]{Notation}
\newtheorem{problem}[theorem]{Problem}

\theoremstyle{definition}
\newtheorem{definition}{\color{MidnightBlue}{\textbf{Definition}}}[section]
\newtheorem{example}{\color{WildStrawberry}Example}[section]
\newtheorem{assumption}{Assumption}[section]
\newtheorem{calculation}{Calculation}[section]
\newtheorem{condition}[assumption]{Condition}
\newtheorem*{solution}{\color{Goldenrod}Solution}
% \newenvironment{solution}[1][\proofname]{%
%   \proof[\bf \color{Goldenrod}Solution to #1]%
% }{\endproof}

\newtheorem{exercise}{\color{YellowOrange}Exercise}[section]

% Literature Summary Standards
\newtheorem*{motivation}{Motivation}
\newtheorem*{summary}{Summary}
\newtheorem*{remark}{Remark}
\newtheorem*{model}{Model}
\newtheorem*{tresults}{Theoretical Results}
\newtheorem*{eresults}{Empirical Results}

%%%%%%%%%%%%%%%%%%%%%%%%%%%%%%%%%%%%%%%%%%%%%%%%%%%%%%%%%%%%%%%%%%%%%%%%%%%%%%%%%%

% Math macros

% Math ``brackets''
\newcommand\parens[1]{\left( #1 \right)}
\newcommand\squares[1]{\left[ #1 \right]}
\newcommand\braces[1]{\left\{ #1 \right\}}
\newcommand\angles[1]{\left\langle #1 \right\rangle}
\newcommand\ceil[1]{\left\lceil #1 \right\rceil}
\newcommand\floor[1]{\left\lfloor #1 \right\rfloor}
\newcommand\abs[1]{\left| #1 \right|}
\newcommand\dabs[1]{\left\| #1 \right\|}
\newcommand\vect[1]{\mathbf{#1}}
\newcommand\closure[1]{\overline{#1}}
\newcommand\pset[1]{\mathcal{P}\left(#1\right)}
\newcommand\inv[1]{#1^{-1}}
\newcommand\norm[1]{\lVert#1\rVert}

% inner product
\providecommand{\inner}[1]{\left\langle{#1}\right\rangle}
% stochastic dominance
\newcommand{\lesd}{\preceq_{\textrm{SD}}}

% Set builder (use \Set ultimately and separate by ;)
\DeclarePairedDelimiterX{\set}[1]{\{}{\}}{\setargs{#1}}
\NewDocumentCommand{\setargs}{>{\SplitArgument{1}{;}}m}
{\setargsaux#1}
\NewDocumentCommand{\setargsaux}{mm}
{\IfNoValueTF{#2}{#1} {#1\nonscript\:\delimsize\vert\allowbreak\nonscript\:\mathopen{}#2}}%
\def\Set{\set*}%

% Shortcut math
\newcommand{\ls}{\leqslant}
\newcommand{\gs}{\geqslant}
\def\ss{\subset}
\def\sse{\subseteq}
\def\nss{\not \ss}
\def\sps{\supset}
\def\pss{\subsetneq}
\def\prece{\preccurlyeq}
\def\condgap{\hspace{1cm}}
\def\eprec{\preceq}
% argmax and min
\newcommand{\argmax}{\operatornamewithlimits{argmax}}
\newcommand{\argmin}{\operatornamewithlimits{argmin}}
\newcommand{\es}{\emptyset}
% Implication and reverse implication
\def\imp{\Rightarrow}
\def\pmi{\Leftarrow}
% Integers up to number
\newcommand\intsfin[1]{\braces{1, \ldots, #1}}
% Logic
\def\bic{\Leftrightarrow}
% Bold and italic
\newcommand\boldit[1]{\textbf{\textit{#1}}}
% Misc math
\newcommand{\st}{\ensuremath{\ \mathrm{s.t.}\ }}
\newcommand{\setntn}[2]{ \{ #1 : #2 \} }
\newcommand{\cf}[1]{ \lstinline|#1| }
\newcommand{\fore}{\therefore \quad}
\newcommand{\tod}{\stackrel { d } {\to} }
\newcommand{\tow}{\stackrel { w } {\to} }
\newcommand{\toprob}{\stackrel { p } {\to} }
\newcommand{\toms}{\stackrel { ms } {\to} }
\newcommand{\eqdist}{\stackrel{d} {=} }
\newcommand{\iidsim}{\stackrel{\textrm{ {\sc iid }}} {\sim} }
\newcommand{\1}{\mathbbm 1}
\newcommand{\dee}{\,{\rm d}}
\newcommand{\given}{\, | \,}
\newcommand{\la}{\langle}
\newcommand{\ra}{\rangle}

% Shortcut greek
\def\a{\alpha}
\def\b{\beta}
\def\g{\gamma}
\def\D{\Delta}
\def\d{\delta}
\def\z{\zeta}
\def\k{\kappa}
\def\l{\lambda}
\def\n{\nu}
\def\r{\rho}
\def\s{\sigma}
\def\t{\tau}
\def\x{\xi}
\def\w{\omega}
\def\W{\Omega}
% Nice greek
\newcommand{\p}{\varphi}
\newcommand{\e}{\varepsilon}

% Shorcut vectors
\def\vx{\vect{x}}
\def\vy{\vect{y}}
\def\va{\vect{a}}
\def\vb{\vect{b}}

\newcommand{\CC}{\mathbb C}
\newcommand{\FF}{\mathbb F}
\newcommand{\RR}{\mathbb R}
\newcommand{\NN}{\mathbb N}
\newcommand{\PP}{\mathbbm P}
\newcommand{\EE}{\mathbbm E}
\newcommand{\TT}{\mathbbm T}
\newcommand{\VV}{\mathbbm V}
\newcommand{\QQ}{\mathbb Q}
\newcommand{\WW}{\mathbbm W}
\newcommand{\ZZ}{\mathbbm Z}
\renewcommand{\SS}{\mathbbm S}

% Expectation/Probability
\newcommand{\ee}[1]{\mathbbm{E}[{#1}]}
\newcommand{\pp}[1]{\mathbbm{P}({#1})}

\newcommand{\GG}{\mathsf G}
\newcommand{\XX}{\mathsf X}
\renewcommand{\AA}{\mathsf A}
\newcommand{\YY}{\mathsf Y}
\newcommand{\ZZZ}{\mathsf Z}

\newcommand{\aA}{\mathscr A}
\newcommand{\iI}{\mathscr I}
\newcommand{\eE}{\mathscr E}
\newcommand{\fF}{\mathscr F}
\newcommand{\rR}{\mathscr R}
\newcommand{\lL}{\mathscr L}
\newcommand{\cG}{\mathscr G}

\newcommand{\pP}{\mathcal P}
\newcommand{\aAA}{\mathcal A}
\newcommand{\vV}{\mathcal V}
\newcommand{\mM}{\mathcal M}
\newcommand{\oO}{\mathcal O}
\newcommand{\gG}{\mathcal G}
\newcommand{\hH}{\mathcal H}
\newcommand{\tT}{\mathcal T}
\newcommand{\bB}{\mathcal B}
\newcommand{\zZ}{\mathcal Z}
\newcommand{\cC}{\mathcal C}
\newcommand{\dD}{\mathcal D}
\newcommand{\wW}{\mathcal W}
\newcommand{\uU}{\mathcal U}
\newcommand{\sS}{\mathcal S}

% Common collections
\def\cA{\col{A}}
\def\cB{\col{B}}
% \def\cC{\col{C}}
\def\cT{\col{T}}
\def\cU{\col{U}}

% Common closures
\def\clA{\closure{A}}
\def\clB{\closure{B}}
\def\clK{\closure{K}}

% operators
\DeclareMathOperator{\cl}{cl}
\DeclareMathOperator{\graph}{graph}
\DeclareMathOperator{\interior}{int}
\DeclareMathOperator{\Prob}{Prob}
\DeclareMathOperator{\determinant}{det}
\DeclareMathOperator{\trace}{trace}
\DeclareMathOperator{\sgn}{sgn}
\DeclareMathOperator{\Span}{span}
\DeclareMathOperator{\diag}{diag}
\DeclareMathOperator{\proj}{proj}
\DeclareMathOperator{\rank}{rank}
\DeclareMathOperator{\cov}{Cov}
\DeclareMathOperator{\corr}{Corr}
\DeclareMathOperator{\var}{Var}
\DeclareMathOperator{\mse}{mse}
\DeclareMathOperator{\se}{se}
\DeclareMathOperator{\row}{row}
\DeclareMathOperator{\col}{col}
\DeclareMathOperator{\range}{rng}
\DeclareMathOperator{\kernel}{ker}
\DeclareMathOperator{\dimension}{dim}
\DeclareMathOperator{\bias}{bias}
\DeclareMathOperator{\dom}{dom}
\DeclareMathOperator{\ran}{ran}
\DeclareMathOperator{\Int}{Int}
\DeclareMathOperator{\Cl}{Cl}
\DeclareMathOperator{\im}{im}
\DeclareMathOperator{\conv}{conv}

% \end{extract*}


% \end{extract*}

\title{Problem Set Econ - Week 3 - DSGE models}
\author{Jeanne Sorin}
\begin{document}
\maketitle



\section*{Exercise 1}
In the Brock-Mirman model, there is only one equation, as there is no labor-leisure choice. The one Euler Equation (EE) is:
\begin{equation}
\frac{1}{e^{z_{t}} K_{t}^{\alpha}-K_{t+1}}=\beta E_{t}\left\{\frac{\alpha e^{z_{t+1}} K_{t+1}^{\alpha-1}}{e^{z_{t+1}} K_{t+1}^{\alpha}-K_{t+2}}\right\}
\end{equation}

This time we want to rewrite the equation as the $\Gamma$ function (using Uhlig's notation) to analytically find the values of the matrices F, G, H, L, M, N and consequently P and Q.
\\
\\
To do so we rewrite the above equation:

\begin{equation}
	E_t \left\{\b \frac{ \a e^{z_{t+1}} K_{t+1}^{\a - 1} (e^{z_t} K_t ^\a -K_{t+1})}{e^{z_{t+1}}K_{t+1}^{\a} - K_{t+2}} -1 \right\}= 0
\end{equation}

This is equivalent to :
\begin{equation}
	E_t \left\{\Gamma (X_{t+1}, X_t, X_{t-1}, Z_{t+1}, Z_t) \right\} = 0
\end{equation}
With $X_{t+1} = K_{t+1}$, $X_{t} = K_{t}$, $X_{t-1} = K_{t-1}$, $Z_{t+1} = Z_{t+1}$, $Z_{t} = Z_{t}$ and our $\Gamma$ function having only 1 dimension.
\\
\\
We differentiate $\Gamma$ wrt all parameters, then evaluate them at the steady state for z == 0. \\
For F:
\begin{align}
	\frac{d\Gamma}{d X_{t+1}} &= \frac{d\Gamma}{d K_{t+2}} = E \left\{\b \frac{\a e^{z_{t+1}} K_{t+1}^{\a - 1} (e^{z_t} K_t ^\a -K_{t+1})}{(e^{z_{t+1}}K_{t+1}^{\a} - K_{t+2})^2}
	\right\} \\
	\frac{d\Gamma}{d X} &= \left\{\b \frac{\a K^{\a - 1} (K ^\a -K)}{(K^{\a} - K)^2}
	\right\} \\
	F &=  \left\{\b \frac{\a K^{\a - 1}}{K_\a. K} \right\}
\end{align}

For G (evaluated at the steady state):
\begin{align}
	\frac{d\Gamma}{d X_{t}} &= \frac{\b (\a K^{\a-1})((\a -1)K^{\a - 1}-\a)(K^\a - K) - \a K^{\a -1} (K^\a - 1)}{(K^\a-K)^2} \\
	\frac{d\Gamma}{d X_{t}} &= \frac{\b (\a K^{\a-1})(K^\a - K)(-K^{\a-1} - \a)}{K^\a - 1)}{(K^\a-K)^2}\\
	\frac{ d\Gamma}{d X_{t}} &= \frac{\b (\a K^{\a-1})(-K^{\a-1} - \a)}{K^\a - 1)}{(K^\a-K)}\\
	G &= -\frac{\b (\a K_{\a-1})(K^{\a-1} + \a)}{K^\a - 1)}{(K^\a-K)}
\end{align}

For H (evaluated at the steady state):
\begin{align}
	\frac{d\Gamma}{d X_{t-1}} &= \frac{\b \a K^{\a-1}\a K^{\a-1}}{K^\a - K}\\
	\frac{d\Gamma}{d X_{t-1}} &= \frac{\b \a^2 K^{2(\a-1)}}{K^\a - K}\\
	H &= \frac{\b \a^2 K^{2(\a-1)}}{K^\a - K}\\
\end{align}

For L (evaluated at the steady state):
\begin{align}
	\frac{d\Gamma}{d Z_{t+1}} &= L = - \frac{\b \a K^{2(\a-1)}}{K^\a - K}\\
\end{align}

For M (evaluated at the steady state):
\begin{align}
	\frac{d\Gamma}{d Z_{t}} &= M = - \frac{\b \a^2 K{2(\a-1)}}{K^\a - K}\\
\end{align}

Moreover, we have
\begin{align}
GP^2 + GP + H = 0 \\
FQN + (FP + G) Q + (LN + M) = 0
\end{align}
Rearranging (see exercise 3 for details)
\begin{align}
H &= \frac{\a^2 K^{2(\a-1)}}{K^\a - K}\\
P &= \frac{-G +/- (G^2 - 4FH)^{0.5}}{2F}\\
Q &= - \frac{LN + M}{FN + FP + G}
\end{align}

\noindent Fitting in our calibrated parameter values from the DSGE lecture: $\b=.98$, $\a=.40$, $z=0$, and having :
\begin{align}
F &=  \left\{- \frac{\a K^{\a - 1}}{K_\a. K} \right\}\\
G &= -\frac{(\a K_{\a-1})(K^{\a-1} + \a)}{K^\a - 1)}{(K^\a-K)}\\
H &= \frac{\a^2 K^{2(\a-1)}}{K^\a - K}\\
L &= - \frac{\a K_{2(\a-1)}}{K^\a - K}\\
M &= - \frac{\a^2 K_{2(\a-1)}}{K^\a - K}\\
P &= \frac{-G \pm (G^2 - 4FH)^{0.5}}{2F}\\
Q &= - \frac{LN + M}{FN + FP + G}
\end{align}
becomes
\begin{align}
F &=  \left\{-\frac{.40 K^{-0.60}}{K_{.40}. K} \right\}\\
G &= -\frac{(\a K_{-0.60})(K^{-0.60} + 0.40)}{K^{.40} - 1)}{(K^{.40}-K)}\\
H &= \frac{0.40^2 K^{-1.2}}{K^({0.40}) - K}\\
L &= - \frac{\a K_{-1.2}}{K^{.40} - K}\\
M &= - \frac{\a^2 K_{-1.2}}{K^{.40} - K}\\
Solve P Q
\end{align}

See Notebook for more the values of P \& Q

\section*{Exercise 2}

Now we redo the exercise with $k = \ln(K)$. Using the previous equations, we can replace $K$ by $e^k$ and take the logarithm of the gamma equation to get:

\begin{equation}
	\ln \parens{E_t \left\{\b \frac{ \a e^{z_{t+1} + (\a - 1)k_{t+1}} ( e^{z_t + \a k_t} - e^{ k_{t+1}})}{e^{z_{t+1} + \a k_{t+1}} - e^{k_{t+2}}} \right\}} = 0 
\end{equation}

\begin{equation}
\ln \b +  \ln \parens{\a e^{z_{t+1} + (\a - 1)k_{t+1} + z_t + \a k_t} - \a e^{z_{t+1} + \a k_{t+1}}} - \ln \parens{e^{z_{t+1} + \a k_{t+1}} - e^{k_{t+2}}} = 0
\end{equation}


We then have the following equations:

\begin{align*}
F &=  \frac{e^{z_{t+1}}}{e^{z_{t+1} + \a k_{t+1}} - e^{k_{t+2}}} \\
G &= \frac{(\a -1)  e^{z_{t+1} + (\a - 1)k_{t+1} + z_t + \a k_t} - \a e^{z_{t+1} + \a k_{t+1}}}{ e^{z_{t+1} + (\a - 1)k_{t+1} + z_t + \a k_t} - e^{z_{t+1} + \a k_{t+1}}} - \frac{\a e^{z_{t+1} + \a k_{t+1}}}{e^{z_{t+1} + \a k_{t+1}} - e^{k_{t+2}}} \\
H &= \frac{\a  e^{z_{t+1} + (\a - 1)k_{t+1} + z_t + \a k_t}}{ e^{z_{t+1} + (\a - 1)k_{t+1} + z_t + \a k_t} - e^{z_{t+1} + \a k_{t+1}}} \\
L &= 1 -  \frac{e^{z_{t+1} + \a k_{t+1}}}{e^{z_{t+1} + \a k_{t+1}} - e^{k_{t+2}}} \\
M &=  \frac{ e^{z_{t+1} + (\a - 1)k_{t+1} + z_t + \a k_t}}{ e^{z_{t+1} + (\a - 1)k_{t+1} + z_t + \a k_t} -  e^{z_{t+1} + \a k_{t+1}}} \\
P &= \frac{-G \pm (G^2 - 4FH)^{\frac{1}{2}}}{2F} \\
Q &= - \frac{LN + M}{FN + FP + G}
\end{align*}



% \listoftodos

\section*{Exercise 3}

\begin{align}
E_{t}\left\{F \tilde{X}_{t+1}+G \tilde{X}_{t}+H \tilde{X}_{t-1}+L \tilde{Z}_{t+1}+M \tilde{Z}_{t}\right\} &= 0\\
E_{t}\left\{F \parens{P \tilde{X}_t + Q \tilde{Z}_{t+1}} +G \parens{P \tilde{X}_{t-1} + Q \tilde{Z}_t}+H \tilde{X}_{t-1}+L \tilde{Z}_{t+1}+M \tilde{Z}_{t}\right\} &= 0 \\
F \squares{P \parens{P \tilde{X}_{t-1} + Q \tilde{Z}_t} + Q N \tilde{Z}_{t}} +G \parens{P \tilde{X}_{t-1} + Q \tilde{Z}_t}+H \tilde{X}_{t-1}+L \tilde{Z}_{t+1}+M \tilde{Z}_{t} &= 0 \\
[(F P+G) P+H] \tilde{X}_{t-1}+[(F Q+L) N+(F P+G) Q+M] \tilde{Z}_{t}&=0
\end{align}


%\listoftodos 
%\begin{align}
%E_{t}\left\{F \tilde{X}_{t+1}} &= F(P \tilde{X}_{t} + Q \tilde{Z}_{t+1}) \right\}\\
%E_{t}\left\{F \tilde{X}_{t+1}} &= F P^2 \tilde{X}_{t-1} + FPQ\tilde{Z}_{t} + FQN \tilde{Z}_{t} \right\}\\
%\end{align}



\end{document}