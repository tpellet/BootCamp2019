 \documentclass[11pt]{article}

% Extract
% \usepackage[active,
%             generate=topology_definitions,
%             %extract-cmd={section},
%             extract-env={definition,algorithm}]{extract}

% \usepackage[active,
%             generate=topology_theorems,
%             %extract-cmd={section},
%             extract-env={theorem,corollary,claim}]{extract}

% \begin{extract*}

%%%%%%%%%%%%%%%%%%%%%%%%%%%%%%%%%%%%%%%%%%%%%%%%%%%%%%%%%%%%%%%%%%%%%%%%%%%%%%%%%%

% Packages

% AMS 
\usepackage{amsmath, amssymb, amsthm, amsbsy}
% Geometry
\usepackage{geometry}
% Colors
\usepackage[usenames,dvipsnames]{xcolor}
% Figures
\usepackage{graphicx}
\usepackage{float}
% Multi column lists
\usepackage{multicol}
% Subfigures
\usepackage{caption}
\usepackage{subcaption}
% Caligraphic
\usepackage{mathrsfs}
\usepackage{bbm}
% Bold
\usepackage{bm}
% algos
\usepackage[linesnumbered, lined, ruled]{algorithm2e}
% Spacing 
\usepackage{setspace}
% Refs/links
\usepackage[colorlinks=true, citecolor=Blue, linkcolor=blue]{hyperref}
\newcommand\myshade{85}
\colorlet{mylinkcolor}{violet}
\colorlet{mycitecolor}{PineGreen}
\colorlet{myurlcolor}{Aquamarine}

\hypersetup{
  linkcolor  = mylinkcolor!\myshade!black,
  citecolor  = mycitecolor!\myshade!black,
  urlcolor   = myurlcolor!\myshade!black,
  colorlinks = true,
}
% Bibliography
\usepackage{filecontents}
\usepackage{natbib}
% Indent
\usepackage{indentfirst}
% Pretty lists
\usepackage{enumitem}
\setlist[enumerate]{itemsep=2pt,topsep=3pt}
\setlist[itemize]{itemsep=2pt,topsep=3pt}
\setlist[enumerate,1]{label=(\roman*)}

% Code
\usepackage{listings}

% Appendix
\usepackage[toc,page]{appendix}

% Math
\usepackage{mathtools}
\usepackage{xparse}

% Equation numbering
\numberwithin{equation}{section}

% Use more than one optional parameter in a new commands
\usepackage{xargs}                      
% Todo
\usepackage[colorinlistoftodos,prependcaption,textsize=normalsize]{todonotes}
\newcommandx{\unsure}[2][1=]{\todo[linecolor=red,backgroundcolor=red!25,bordercolor=red,#1]{#2}}
\newcommandx{\change}[2][1=]{\todo[linecolor=blue,backgroundcolor=blue!25,bordercolor=blue,#1]{#2}}
\newcommandx{\info}[2][1=]{\todo[linecolor=OliveGreen,backgroundcolor=OliveGreen!25,bordercolor=OliveGreen,#1]{#2}}
\newcommandx{\improvement}[2][1=]{\todo[linecolor=Plum,backgroundcolor=Plum!25,bordercolor=Plum,#1]{#2}}
\newcommandx{\thiswillnotshow}[2][1=]{\todo[disable,#1]{#2}}

%%%%%%%%%%%%%%%%%%%%%%%%%%%%%%%%%%%%%%%%%%%%%%%%%%%%%%%%%%%%%%%%%%%%%%%%%%%%%%%%%%

% Document Settings

% Figure path
\graphicspath{{./figures/}}
% Matrix columns
\setcounter{MaxMatrixCols}{10}
% So pages will break inside long equation environments
\allowdisplaybreaks
% Font
\usepackage{mathpazo} 
\linespread{1.05}  
%\usepackage{courier}
% Geometry
\geometry{left=1in,right=1in,top=1in,bottom=1in}
% Counters
\setcounter{tocdepth}{2}
\setcounter{secnumdepth}{3}

%%%%%%%%%%%%%%%%%%%%%%%%%%%%%%%%%%%%%%%%%%%%%%%%%%%%%%%%%%%%%%%%%%%%%%%%%%%%%%%%%%

% Colors

\definecolor{Tm}{rgb}{0,0,0.80}
\newcommand{\defi}[1]{\textcolor{MidnightBlue}{\bf #1}}

%%%%%%%%%%%%%%%%%%%%%%%%%%%%%%%%%%%%%%%%%%%%%%%%%%%%%%%%%%%%%%%%%%%%%%%%%%%%%%%%%%

% Environments

\theoremstyle{plain}
\newtheorem{theorem}{\color{ForestGreen}{\textbf{Theorem}}}[section]
\newtheorem{claim}{\color{ForestGreen}{\textbf{Claim}}}[section]
\newtheorem{lemma}[theorem]{\color{ForestGreen}{\textbf{Lemma}}}
\newtheorem{proposition}[theorem]{\color{ForestGreen}{\textbf{Proposition}}}
\newtheorem{corollary}[theorem]{\color{ForestGreen}{\textbf{Corollary}}}
\newtheorem{axiom}[theorem]{\color{ForestGreen}{\textbf{Axiom}}}
\newtheorem{conjecture}[theorem]{Conjecture}
\newtheorem{case}[theorem]{Case}
\newtheorem{conclusion}[theorem]{Conclusion}
\newtheorem{criterion}[theorem]{Criterion}
\newtheorem{notation}[theorem]{Notation}
\newtheorem{problem}[theorem]{Problem}

\theoremstyle{definition}
\newtheorem{definition}{\color{MidnightBlue}{\textbf{Definition}}}[section]
\newtheorem{example}{\color{WildStrawberry}Example}[section]
\newtheorem{assumption}{Assumption}[section]
\newtheorem{calculation}{Calculation}[section]
\newtheorem{condition}[assumption]{Condition}
\newtheorem*{solution}{\color{Goldenrod}Solution}
% \newenvironment{solution}[1][\proofname]{%
%   \proof[\bf \color{Goldenrod}Solution to #1]%
% }{\endproof}

\newtheorem{exercise}{\color{YellowOrange}Exercise}[section]

% Literature Summary Standards
\newtheorem*{motivation}{Motivation}
\newtheorem*{summary}{Summary}
\newtheorem*{remark}{Remark}
\newtheorem*{model}{Model}
\newtheorem*{tresults}{Theoretical Results}
\newtheorem*{eresults}{Empirical Results}

% % Framed theorems below (need to redefine each environment)

% \usepackage{mdframed}

% % Example
% \newmdtheoremenv{theorem}{\color{ForestGreen}{\textbf{Theorem}}}[section]



%%%%%%%%%%%%%%%%%%%%%%%%%%%%%%%%%%%%%%%%%%%%%%%%%%%%%%%%%%%%%%%%%%%%%%%%%%%%%%%%%%

% Math macros

% Math ``brackets''
\newcommand\parens[1]{\left( #1 \right)}
\newcommand\squares[1]{\left[ #1 \right]}
\newcommand\braces[1]{\left\{ #1 \right\}}
\newcommand\angles[1]{\left\langle #1 \right\rangle}
\newcommand\ceil[1]{\left\lceil #1 \right\rceil}
\newcommand\floor[1]{\left\lfloor #1 \right\rfloor}
\newcommand\abs[1]{\left| #1 \right|}
\newcommand\dabs[1]{\left\| #1 \right\|}
\newcommand\vect[1]{\mathbf{#1}}
\newcommand\closure[1]{\overline{#1}}
\newcommand\pset[1]{\mathcal{P}\left(#1\right)}
\newcommand\inv[1]{#1^{-1}}
\newcommand\norm[1]{\lVert#1\rVert}

% inner product
\providecommand{\inner}[1]{\left\langle{#1}\right\rangle}
% stochastic dominance
\newcommand{\lesd}{\preceq_{\textrm{SD}}}

% Set builder (use \Set ultimately and separate by ;)
\DeclarePairedDelimiterX{\set}[1]{\{}{\}}{\setargs{#1}}
\NewDocumentCommand{\setargs}{>{\SplitArgument{1}{;}}m}
{\setargsaux#1}
\NewDocumentCommand{\setargsaux}{mm}
{\IfNoValueTF{#2}{#1} {#1\nonscript\:\delimsize\vert\allowbreak\nonscript\:\mathopen{}#2}}%
\def\Set{\set*}%

% Shortcut math
\newcommand{\ls}{\leqslant}
\newcommand{\gs}{\geqslant}
\def\ss{\subset}
\def\sse{\subseteq}
\def\nss{\not \ss}
\def\sps{\supset}
\def\pss{\subsetneq}
\def\prece{\preccurlyeq}
\def\condgap{\hspace{1cm}}
\def\eprec{\preceq}
% argmax and min
\newcommand{\argmax}{\operatornamewithlimits{argmax}}
\newcommand{\argmin}{\operatornamewithlimits{argmin}}
\newcommand{\es}{\emptyset}
% Implication and reverse implication
\def\imp{\Rightarrow}
\def\pmi{\Leftarrow}
% Integers up to number
\newcommand\intsfin[1]{\braces{1, \ldots, #1}}
% Logic
\def\bic{\Leftrightarrow}
% Bold and italic
\newcommand\boldit[1]{\textbf{\textit{#1}}}
% Misc math
\newcommand{\st}{\ensuremath{\ \mathrm{s.t.}\ }}
\newcommand{\setntn}[2]{ \{ #1 : #2 \} }
\newcommand{\cf}[1]{ \lstinline|#1| }
\newcommand{\fore}{\therefore \quad}
\newcommand{\tod}{\stackrel { d } {\to} }
\newcommand{\tow}{\stackrel { w } {\to} }
\newcommand{\toprob}{\stackrel { p } {\to} }
\newcommand{\toms}{\stackrel { ms } {\to} }
\newcommand{\eqdist}{\stackrel{d} {=} }
\newcommand{\iidsim}{\stackrel{\textrm{ {\sc iid }}} {\sim} }
\newcommand{\1}{\mathbbm 1}
\newcommand{\dee}{\,{\rm d}}
\newcommand{\given}{\, | \,}
\newcommand{\la}{\langle}
\newcommand{\ra}{\rangle}

% Shortcut greek
\def\a{\alpha}
\def\b{\beta}
\def\g{\gamma}
\def\D{\Delta}
\def\d{\delta}
\def\z{\zeta}
\def\k{\kappa}
\def\l{\lambda}
\def\n{\nu}
\def\r{\rho}
\def\s{\sigma}
\def\t{\tau}
\def\x{\xi}
\def\w{\omega}
\def\W{\Omega}
% Nice greek
\newcommand{\p}{\varphi}
\newcommand{\e}{\varepsilon}

% Shorcut vectors
\def\vx{\vect{x}}
\def\vy{\vect{y}}
\def\va{\vect{a}}
\def\vb{\vect{b}}

\newcommand{\CC}{\mathbb C}
\newcommand{\FF}{\mathbb F}
\newcommand{\RR}{\mathbb R}
\newcommand{\NN}{\mathbb N}
\newcommand{\PP}{\mathbbm P}
\newcommand{\EE}{\mathbbm E}
\newcommand{\TT}{\mathbbm T}
\newcommand{\VV}{\mathbbm V}
\newcommand{\QQ}{\mathbb Q}
\newcommand{\WW}{\mathbbm W}
\newcommand{\ZZ}{\mathbbm Z}
\renewcommand{\SS}{\mathbbm S}

% Expectation/Probability
\newcommand{\ee}[1]{\mathbbm{E}[{#1}]}
\newcommand{\pp}[1]{\mathbbm{P}({#1})}

\newcommand{\GG}{\mathsf G}
\newcommand{\XX}{\mathsf X}
\renewcommand{\AA}{\mathsf A}
\newcommand{\YY}{\mathsf Y}
\newcommand{\ZZZ}{\mathsf Z}

\newcommand{\aA}{\mathscr A}
\newcommand{\iI}{\mathscr I}
\newcommand{\eE}{\mathscr E}
\newcommand{\fF}{\mathscr F}
\newcommand{\rR}{\mathscr R}
\newcommand{\lL}{\mathscr L}
\newcommand{\cG}{\mathscr G}

\newcommand{\pP}{\mathcal P}
\newcommand{\aAA}{\mathcal A}
\newcommand{\vV}{\mathcal V}
\newcommand{\mM}{\mathcal M}
\newcommand{\oO}{\mathcal O}
\newcommand{\gG}{\mathcal G}
\newcommand{\hH}{\mathcal H}
\newcommand{\tT}{\mathcal T}
\newcommand{\bB}{\mathcal B}
\newcommand{\zZ}{\mathcal Z}
\newcommand{\cC}{\mathcal C}
\newcommand{\dD}{\mathcal D}
\newcommand{\wW}{\mathcal W}
\newcommand{\uU}{\mathcal U}
\newcommand{\sS}{\mathcal S}

% Common collections
\def\cA{\col{A}}
\def\cB{\col{B}}
% \def\cC{\col{C}}
\def\cT{\col{T}}
\def\cU{\col{U}}

% Common closures
\def\clA{\closure{A}}
\def\clB{\closure{B}}
\def\clK{\closure{K}}

% operators
\DeclareMathOperator{\cl}{cl}
\DeclareMathOperator{\graph}{graph}
\DeclareMathOperator{\interior}{int}
\DeclareMathOperator{\Prob}{Prob}
\DeclareMathOperator{\determinant}{det}
\DeclareMathOperator{\trace}{trace}
\DeclareMathOperator{\sgn}{sgn}
\DeclareMathOperator{\Span}{span}
\DeclareMathOperator{\diag}{diag}
\DeclareMathOperator{\proj}{proj}
\DeclareMathOperator{\rank}{rank}
\DeclareMathOperator{\cov}{Cov}
\DeclareMathOperator{\corr}{Corr}
\DeclareMathOperator{\var}{Var}
\DeclareMathOperator{\mse}{mse}
\DeclareMathOperator{\se}{se}
\DeclareMathOperator{\row}{row}
\DeclareMathOperator{\col}{col}
\DeclareMathOperator{\range}{rng}
\DeclareMathOperator{\kernel}{ker}
\DeclareMathOperator{\dimension}{dim}
\DeclareMathOperator{\bias}{bias}
\DeclareMathOperator{\dom}{dom}
\DeclareMathOperator{\ran}{ran}
\DeclareMathOperator{\Int}{Int}
\DeclareMathOperator{\Cl}{Cl}
\DeclareMathOperator{\im}{im}
\DeclareMathOperator{\conv}{conv}

% \end{extract*}

\title{Problem set 1 - Week 3 - DSGE models}
\author{Thomas Pellet}
\begin{document}
\maketitle



\section*{Exercise 1}

The Euler equation of the Brock-Mirman model is:
\begin{equation}
\frac{1}{e^{z_{t}} K_{t}^{\alpha}-K_{t+1}}=\beta E_{t}\left\{\frac{\alpha e^{z_{t+1}} K_{t+1}^{\alpha-1}}{e^{z_{t+1}} K_{t+1}^{\alpha}-K_{t+2}}\right\}
\end{equation}

We guess the policy function to be $ K_{t+1}=A e^{z_{t}} K_{t}^{\alpha} $ and replace $K_{t+2}$ in the Euler equation. We then have:

\begin{align}
\frac{1}{\frac{1}{A}K_{t+1}-K_{t+1}}&=\beta E_{t}\left\{\frac{\alpha e^{z_{t+1}} K_{t+1}^{\alpha-1}}{e^{z_{t+1}} K_{t+1}^{\alpha}-A e^{z_{t+1}} K_{t+1}^{\alpha}}\right\} \\
\bic A^2 - (\b \a + 1) A - \b \a &= 0
\end{align}

The two solutions to this equation are ${\a \b, 1}$.

\section*{Exercise 2}

The functional forms of the utility and the production functions are given by:

\begin{align} 
u\left(c_{t}, \ell_{t}\right) &=\ln c_{t}+a \ln \left(1-\ell_{t}\right) \\ 
F\left(K_{t}, L_{t}, z_{t}\right) &=e^{z_{t}} K_{t}^{\alpha} L_{t}^{1-\alpha} 
\end{align}

We can replace these in the baseline model to get the following seven equations:
\begin{align}
c_{t}&=(1-\tau)\left[w_{t} \ell_{t}+\left(r_{t}-\delta\right) k_{t}\right]+k_{t}+T_{t}-k_{t+1} \\
\frac{1}{c_t}&=\beta E_{t}\left\{\frac{1}{c_{t+1}}\left[\left(r_{t+1}-\delta\right)(1-\tau)+1\right]\right\} \\
-\frac{a}{1-l_t}&=\frac{1}{c_t} w_{t}(1-\tau) \\
r_{t}&=\a e^{z_{t}} K_{t}^{\alpha-1} L_{t}^{1-\alpha} \\
w_{t}&=(1 - \a) e^{z_{t}} K_{t}^{\alpha} L_{t}^{-\alpha} \\
\tau\left[w_{t} \ell_{t}+\left(r_{t}-\delta\right) k_{t}\right]&=T_{t}
\end{align}

We cannot use the same technic as in the first exercise because we know have consumption and leisure. The random term does not cancel out if we guess a functional form, making the integral impossible to solve analytically. 


\section{Exercise 3}
The functional forms are now 

\begin{align} 
u\left(c_{t}, \ell_{t}\right) &=\frac{c_{t}^{1-\gamma}-1}{1-\gamma}+a \ln \left(1-\ell_{t}\right) \\ 
F\left(K_{t}, L_{t}, z_{t}\right) &=e^{z_{t}} K_{t}^{\alpha} L_{t}^{1-\alpha} \end{align}

and the characterizing equations are now:

\begin{align}
c_{t}&=(1-\tau)\left[w_{t} \ell_{t}+\left(r_{t}-\delta\right) k_{t}\right]+k_{t}+T_{t}-k_{t+1} \\
c_t^{-\g}&=\beta E_{t}\left\{c_{t+1}^{-\g}\left[\left(r_{t+1}-\delta\right)(1-\tau)+1\right]\right\} \\
-\frac{a}{1-l_t}&=c_t^{-\g} w_{t}(1-\tau) \\
r_{t}&=\a e^{z_{t}} K_{t}^{\alpha-1} L_{t}^{1-\alpha} \\
w_{t}&=(1 - \a) e^{z_{t}} K_{t}^{\alpha} L_{t}^{-\alpha} \\
\tau\left[w_{t} \ell_{t}+\left(r_{t}-\delta\right) k_{t}\right]&=T_{t}
\end{align}

\section{Exercise 4}
The functional forms are now 

\begin{align} 
 u\left(c_{t}, \ell_{t}\right) &=\frac{c_{t}^{1-\gamma}-1}{1-\gamma}+a \frac{\left(1-\ell_{t}\right)^{1-\xi}-1}{1-\xi} \\ F\left(K_{t}, L_{t}, z_{t}\right) &=e^{z_{t}}\left[\alpha K_{t}^{\eta}+(1-\alpha) L_{t}^{\eta}\right]^{\frac{1}{\eta}} 
 \end{align}

and the characterizing equations are now:

\begin{align}
c_{t}&=(1-\tau)\left[w_{t} \ell_{t}+\left(r_{t}-\delta\right) k_{t}\right]+k_{t}+T_{t}-k_{t+1} \\
c_t^{-\g}&=\beta E_{t}\left\{c_{t+1}^{-\g}\left[\left(r_{t+1}-\delta\right)(1-\tau)+1\right]\right\} \\
\frac{a}{(1-\ell_t)^{\xi}}&=c_t^{-\g} w_{t}(1-\tau) \\
r_{t}&=\a K_{t}^{\eta - 1} e^{z_{t}} \left[\alpha K_{t}^{\eta}+(1-\alpha) L_{t}^{\eta}\right]^{\frac{1}{\eta} - 1} \\
w_{t}&=(1-\a) L_{t}^{\eta - 1} e^{z_{t}} \left[\alpha K_{t}^{\eta}+(1-\alpha) L_{t}^{\eta}\right]^{\frac{1}{\eta} - 1} \\
\tau\left[w_{t} \ell_{t}+\left(r_{t}-\delta\right) k_{t}\right]&=T_{t}
\end{align}


\section{Exercise 5}
The functional forms are now 

\begin{align} 
 u\left(c_{t}, \ell_{t}\right) &=\frac{c_{t}^{1-\gamma}-1}{1-\gamma} \\ 
 F\left(K_{t}, L_{t}, z_{t}\right) &= K_{t}^{\alpha} \parens{e^{z_{t}} L_{t}}^{1-\alpha} 
 \end{align}

and the characterizing equations are now:

\begin{align}
c_{t}&=(1-\tau)\left[w_{t} \ell_{t}+\left(r_{t}-\delta\right) k_{t}\right]+k_{t}+T_{t}-k_{t+1} \\
c_t^{-\g}&=\beta E_{t}\left\{c_{t+1}^{-\g}\left[\left(r_{t+1}-\delta\right)(1-\tau)+1\right]\right\} \\
1&=c_t^{-\g} w_{t}(1-\tau) \\
r_{t}&=\a K_{t}^{\alpha - 1} \parens{e^{z_{t}}}^{1-\alpha} \\
w_{t}&=(1-\a)K_{t}^{\alpha} e^{(1- \alpha) z_{t}} \\
\tau\left[w_{t} \ell_{t}+\left(r_{t}-\delta\right) k_{t}\right]&=T_t \\
z_{t}&=\left(1-\rho_{z}\right) \overline{z}+\rho_{z} z_{t-1}+
\epsilon_{t}^{z}
\end{align}


Solving the model at the steady state, we have:

\begin{align}
c&=(1-\tau)\left[w+\left(r-\delta\right) k\right]+T \\
1&=\beta \left\{\left[\left(r-\delta\right)(1-\tau)+1\right]\right\} \\
1&=c_t^{-\g} w(1-\tau) \\
r&=\a K^{\alpha - 1} \parens{e^{z}}^{1-\alpha} \\
w&=(1-\a)K^{\alpha} e^{(1- \alpha) z} \\
\tau\left[w+\left(r-\delta\right) k\right]&=T_t \\
z^{*}&=\overline{z}
\end{align}

Simplyfying, we get:

\begin{align}
c&= \parens{(1 - \t)(1-\a) \parens{\frac{1-\b}{\b (1 - \t) \a} + \d}^{\frac{1}{\a - 1}}}^{\frac{1}{\g}} \\
r&= \frac{1-\b}{\b (1 - \t)} + \d \\
w&=(1-\a)\parens{\frac{1-\b}{\b (1 - \t) \a} + \d}^{\frac{1}{\a - 1}} \\
k&=\parens{\frac{1-\b}{\b (1 - \t) \a} + \d}^{\frac{1}{\a - 1}} \\
T &=\tau\left[w+\left(r-\delta\right) k\right]\\
z&=\overline{z}
\end{align}

\section{Exercise 6}

The functional forms are now 

\begin{align} 
 u\left(c_{t}, \ell_{t}\right) &=\frac{c_{t}^{1-\gamma}-1}{1-\gamma}+a \frac{\left(1-\ell_{t}\right)^{1-\xi}-1}{1-\xi} \\ 
  F\left(K_{t}, L_{t}, z_{t}\right) &= K_{t}^{\alpha} \parens{e^{z_{t}} L_{t}}^{1-\alpha} 
 \end{align}
and the characterizing equations are now:

\begin{align}
c_{t}&=(1-\tau)\left[w_{t} \ell_{t}+\left(r_{t}-\delta\right) k_{t}\right]+k_{t}+T_{t}-k_{t+1} \\
c_t^{-\g}&=\beta E_{t}\left\{c_{t+1}^{-\g}\left[\left(r_{t+1}-\delta\right)(1-\tau)+1\right]\right\} \\
\frac{a}{(1-\ell_t)^{\xi}}&=c_t^{-\g} w_{t}(1-\tau) \\
r_{t}&=\a K_{t}^{\alpha - 1} e^{(1-\alpha)z_{t}}L_t^{1-\a}\\
w_{t}&=(1-\a)K_{t}^{\alpha} e^{(1- \alpha) z_{t}}L_t^{1-\a} \\
\tau\left[w_{t} \ell_{t}+\left(r_{t}-\delta\right) k_{t}\right]&=T_t
\end{align}

Solving for the steady state, we get the following equations:

\begin{align}
c&=(1-\tau)\left[w \ell+\left(r-\delta\right) k\right]+T \\
1&=\beta E\left\{\left[\left(r-\delta\right)(1-\tau)+1\right]\right\} \\
\frac{a}{(1-\ell)^{\xi}}&=c^{-\g} w(1-\tau) \\
r&=\a K^{\alpha - 1} e^{(1-\alpha)z}L^{1-\a}\\
w&=(1-\a)K^{\alpha} e^{(1- \alpha) z}L^{1-\a} \\
\tau\left[w \ell+\left(r-\delta\right) k\right]&=T
\end{align}


and therefore:

\begin{align}
c&=w \ell+ (r-\delta) k \\
r&= \frac{1-\b}{\b (1 - \t)} + \d \\
r&=\a K^{\alpha - 1} e^{(1-\alpha)z}L^{1-\a}\\
K &= \frac{\a w}{(1-\a)r}\\
c &= \parens{-\frac{w}{a}(1-\ell)^{-\xi}(1-\tau)}^{\frac{1}{\g}}
\end{align}


\section*{Exercise 7}
The model is now defined by:
\begin{align*}
\overline{c}&=(1-\tau)[\overline{w} \overline{\ell}+(\overline{r}-\delta) \overline{k}]+\overline{T} \\ 
u_{c}(\overline{c}, \overline{\ell})&=\beta E_{t}\left\{u_{c}(\overline{c}, \overline{\ell})[(\overline{r}-\delta)(1-\tau)+1]\right\} \\ 
-u_{\ell}(\overline{c}, \overline{\ell})&=u_{c}(\overline{c}, \overline{\ell}) \overline{w}(1-\tau) \\ 
\overline{r}&=f_{K}(\overline{k}, \overline{\ell}, \overline{z}) \\ 
\overline{w}&=f_{L}(\overline{k}, \overline{\ell}, \overline{z}) \\ 
\tau[\overline{w} \overline{\ell}+(\overline{r}-\delta) \overline{k}]&=\overline{T}
\end{align*}

For the numerical differentiation, we need to express every variable as a function of the parameters and other variables of the system:

\begin{align*}
\overline{k} &= \parens{\frac{\overline{r}}{\a} e^{(1-\alpha)z}L^{\a-1}}^{\frac{1}{\alpha - 1}} \\
\overline{\ell} &= 1 - \parens{\frac{\overline{c}^{\g}}{\overline{w}(1-\tau)}}^{\frac{1}{\xi}}  \\
\overline{y} &= F(\overline{k},\overline{l},\overline{z}) \\
\overline{w}&=f_{L}(\overline{k}, \overline{\ell}, \overline{z}) \\
\overline{r}&= \frac{1-\b}{\b (1 - \t)} + \d \\ 
\overline{T}&=\tau[\overline{w} \overline{\ell}+(\overline{r}-\delta) \overline{k}] \\
\overline{c}&=(1-\tau)[\overline{w} \overline{\ell}+(\overline{r}-\delta) \overline{k}]+\overline{T}
\end{align*}




















\newpage


\begin{equation}
	\floor{\squares{\inv{\parens{\parens{\frac{1}{2}} \begin{bmatrix} 3 & 4 \\ 4 & 5\end{bmatrix} }}}}
\end{equation}

\improvement[inline]{Unfinished.}

\begin{theorem}[Test Theorem]
	Hello.
\end{theorem}

\begin{definition}[Test Definition]
	Hello.
\end{definition}

\begin{example}[Test Examples]
	My examples are pink-ish.
\end{example}

\begin{equation}
	\ee{X_{i,j}} \in \RR \quad \pp{X_{i,j}} \in \RR
\end{equation}

\change[inline]{Better math.}

\unsure[inline]{Rebekah is amazing!!!!!!}
% \improvement{Fix this}

\begin{equation}
\sum_{n=1}^{N} \sum_{t=1}^{T} \sum_{j=1}^{J} d_{n j t}\left\{\log \left[p_{j t}\left(x_{n t}\right)\right]+\sum_{x=1}^{X} /\left\{x_{n, t+1}=x\right\} \log \left[f_{j t}\left(x | x_{n t}\right)\right]\right\}
\end{equation}

\begin{eresults}
test
\end{eresults}

$\intsfin{5} \toprob{5} \vx \cA$

% $\Set{\frac{1}{2}}$

\newpage 
\listoftodos

\end{document}