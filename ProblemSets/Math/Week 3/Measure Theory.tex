 \documentclass[11pt]{article}

% Extract
% \usepackage[active,
%             generate=topology_definitions,
%             %extract-cmd={section},
%             extract-env={definition,algorithm}]{extract}

% \usepackage[active,
%             generate=topology_theorems,
%             %extract-cmd={section},
%             extract-env={theorem,corollary,claim}]{extract}

% \begin{extract*}

%%%%%%%%%%%%%%%%%%%%%%%%%%%%%%%%%%%%%%%%%%%%%%%%%%%%%%%%%%%%%%%%%%%%%%%%%%%%%%%%%%

% Packages

% AMS 
\usepackage{amsmath, amssymb, amsthm, amsbsy}
% Geometry
\usepackage{geometry}
% Colors
\usepackage[usenames,dvipsnames]{xcolor}
% Figures
\usepackage{graphicx}
\usepackage{float}
% Multi column lists
\usepackage{multicol}
% Subfigures
\usepackage{caption}
\usepackage{subcaption}
% Caligraphic
\usepackage{mathrsfs}
\usepackage{bbm}
% Bold
\usepackage{bm}
% algos
\usepackage[linesnumbered, lined, ruled]{algorithm2e}
% Spacing 
\usepackage{setspace}
% Refs/links
\usepackage[colorlinks=true, citecolor=Blue, linkcolor=blue]{hyperref}
\newcommand\myshade{85}
\colorlet{mylinkcolor}{violet}
\colorlet{mycitecolor}{PineGreen}
\colorlet{myurlcolor}{Aquamarine}

\hypersetup{
  linkcolor  = mylinkcolor!\myshade!black,
  citecolor  = mycitecolor!\myshade!black,
  urlcolor   = myurlcolor!\myshade!black,
  colorlinks = true,
}
% Bibliography
\usepackage{filecontents}
\usepackage{natbib}
% Indent
\usepackage{indentfirst}
% Pretty lists
\usepackage{enumitem}
\setlist[enumerate]{itemsep=2pt,topsep=3pt}
\setlist[itemize]{itemsep=2pt,topsep=3pt}
\setlist[enumerate,1]{label=(\roman*)}

% Code
\usepackage{listings}

% Appendix
\usepackage[toc,page]{appendix}

% Math
\usepackage{mathtools}
\usepackage{xparse}

% Equation numbering
\numberwithin{equation}{section}

% Use more than one optional parameter in a new commands
\usepackage{xargs}                      
% Todo
\usepackage[colorinlistoftodos,prependcaption,textsize=normalsize]{todonotes}
\newcommandx{\unsure}[2][1=]{\todo[linecolor=red,backgroundcolor=red!25,bordercolor=red,#1]{#2}}
\newcommandx{\change}[2][1=]{\todo[linecolor=blue,backgroundcolor=blue!25,bordercolor=blue,#1]{#2}}
\newcommandx{\info}[2][1=]{\todo[linecolor=OliveGreen,backgroundcolor=OliveGreen!25,bordercolor=OliveGreen,#1]{#2}}
\newcommandx{\improvement}[2][1=]{\todo[linecolor=Plum,backgroundcolor=Plum!25,bordercolor=Plum,#1]{#2}}
\newcommandx{\thiswillnotshow}[2][1=]{\todo[disable,#1]{#2}}

%%%%%%%%%%%%%%%%%%%%%%%%%%%%%%%%%%%%%%%%%%%%%%%%%%%%%%%%%%%%%%%%%%%%%%%%%%%%%%%%%%

% Document Settings

% Figure path
\graphicspath{{./figures/}}
% Matrix columns
\setcounter{MaxMatrixCols}{10}
% So pages will break inside long equation environments
\allowdisplaybreaks
% Font
%\usepackage{mathpazo} 
\linespread{1.05}  
%\usepackage{courier}
% Geometry
\geometry{left=1in,right=1in,top=1in,bottom=1in}
% Counters
\setcounter{tocdepth}{2}
\setcounter{secnumdepth}{3}

%%%%%%%%%%%%%%%%%%%%%%%%%%%%%%%%%%%%%%%%%%%%%%%%%%%%%%%%%%%%%%%%%%%%%%%%%%%%%%%%%%

% Colors

\definecolor{Tm}{rgb}{0,0,0.80}
\newcommand{\defi}[1]{\textcolor{MidnightBlue}{\bf #1}}

%%%%%%%%%%%%%%%%%%%%%%%%%%%%%%%%%%%%%%%%%%%%%%%%%%%%%%%%%%%%%%%%%%%%%%%%%%%%%%%%%%

% Environments

\theoremstyle{plain}
\newtheorem{theorem}{\color{ForestGreen}{\textbf{Theorem}}}[section]
\newtheorem{claim}{\color{ForestGreen}{\textbf{Claim}}}[section]
\newtheorem{lemma}[theorem]{\color{ForestGreen}{\textbf{Lemma}}}
\newtheorem{proposition}[theorem]{\color{ForestGreen}{\textbf{Proposition}}}
\newtheorem{corollary}[theorem]{\color{ForestGreen}{\textbf{Corollary}}}
\newtheorem{axiom}[theorem]{\color{ForestGreen}{\textbf{Axiom}}}
\newtheorem{conjecture}[theorem]{Conjecture}
\newtheorem{case}[theorem]{Case}
\newtheorem{conclusion}[theorem]{Conclusion}
\newtheorem{criterion}[theorem]{Criterion}
\newtheorem{notation}[theorem]{Notation}
\newtheorem{problem}[theorem]{Problem}

\theoremstyle{definition}
\newtheorem{definition}{\color{MidnightBlue}{\textbf{Definition}}}[section]
\newtheorem{example}{\color{WildStrawberry}Example}[section]
\newtheorem{assumption}{Assumption}[section]
\newtheorem{calculation}{Calculation}[section]
\newtheorem{condition}[assumption]{Condition}
\newtheorem*{solution}{\color{Goldenrod}Solution}
% \newenvironment{solution}[1][\proofname]{%
%   \proof[\bf \color{Goldenrod}Solution to #1]%
% }{\endproof}

\newtheorem{exercise}{\color{YellowOrange}Exercise}[section]

% Literature Summary Standards
\newtheorem*{motivation}{Motivation}
\newtheorem*{summary}{Summary}
\newtheorem*{remark}{Remark}
\newtheorem*{model}{Model}
\newtheorem*{tresults}{Theoretical Results}
\newtheorem*{eresults}{Empirical Results}

% % Framed theorems below (need to redefine each environment)

% \usepackage{mdframed}

% % Example
% \newmdtheoremenv{theorem}{\color{ForestGreen}{\textbf{Theorem}}}[section]



%%%%%%%%%%%%%%%%%%%%%%%%%%%%%%%%%%%%%%%%%%%%%%%%%%%%%%%%%%%%%%%%%%%%%%%%%%%%%%%%%%

% Math macros

% Math ``brackets''
\newcommand\parens[1]{\left( #1 \right)}
\newcommand\squares[1]{\left[ #1 \right]}
\newcommand\braces[1]{\left\{ #1 \right\}}
\newcommand\angles[1]{\left\langle #1 \right\rangle}
\newcommand\ceil[1]{\left\lceil #1 \right\rceil}
\newcommand\floor[1]{\left\lfloor #1 \right\rfloor}
\newcommand\abs[1]{\left| #1 \right|}
\newcommand\dabs[1]{\left\| #1 \right\|}
\newcommand\vect[1]{\mathbf{#1}}
\newcommand\closure[1]{\overline{#1}}
\newcommand\pset[1]{\mathcal{P}\left(#1\right)}
\newcommand\inv[1]{#1^{-1}}
\newcommand\norm[1]{\lVert#1\rVert}

% inner product
\providecommand{\inner}[1]{\left\langle{#1}\right\rangle}
% stochastic dominance
\newcommand{\lesd}{\preceq_{\textrm{SD}}}

% Set builder (use \Set ultimately and separate by ;)
\DeclarePairedDelimiterX{\set}[1]{\{}{\}}{\setargs{#1}}
\NewDocumentCommand{\setargs}{>{\SplitArgument{1}{;}}m}
{\setargsaux#1}
\NewDocumentCommand{\setargsaux}{mm}
{\IfNoValueTF{#2}{#1} {#1\nonscript\:\delimsize\vert\allowbreak\nonscript\:\mathopen{}#2}}%
\def\Set{\set*}%

% Shortcut math
\newcommand{\ls}{\leqslant}
\newcommand{\gs}{\geqslant}
\def\ss{\subset}
\def\sse{\subseteq}
\def\nss{\not \ss}
\def\sps{\supset}
\def\pss{\subsetneq}
\def\prece{\preccurlyeq}
\def\condgap{\hspace{1cm}}
\def\eprec{\preceq}
% argmax and min
\newcommand{\argmax}{\operatornamewithlimits{argmax}}
\newcommand{\argmin}{\operatornamewithlimits{argmin}}
\newcommand{\es}{\emptyset}
% Implication and reverse implication
\def\imp{\Rightarrow}
\def\pmi{\Leftarrow}
% Integers up to number
\newcommand\intsfin[1]{\braces{1, \ldots, #1}}
% Logic
\def\bic{\Leftrightarrow}
% Bold and italic
\newcommand\boldit[1]{\textbf{\textit{#1}}}
% Misc math
\newcommand{\st}{\ensuremath{\ \mathrm{s.t.}\ }}
\newcommand{\setntn}[2]{ \{ #1 : #2 \} }
\newcommand{\cf}[1]{ \lstinline|#1| }
\newcommand{\fore}{\therefore \quad}
\newcommand{\tod}{\stackrel { d } {\to} }
\newcommand{\tow}{\stackrel { w } {\to} }
\newcommand{\toprob}{\stackrel { p } {\to} }
\newcommand{\toms}{\stackrel { ms } {\to} }
\newcommand{\eqdist}{\stackrel{d} {=} }
\newcommand{\iidsim}{\stackrel{\textrm{ {\sc iid }}} {\sim} }
\newcommand{\1}{\mathbbm 1}
\newcommand{\dee}{\,{\rm d}}
\newcommand{\given}{\, | \,}
\newcommand{\la}{\langle}
\newcommand{\ra}{\rangle}

% Shortcut greek
\def\a{\alpha}
\def\b{\beta}
\def\g{\gamma}
\def\D{\Delta}
\def\d{\delta}
\def\z{\zeta}
\def\k{\kappa}
\def\l{\lambda}
\def\n{\nu}
\def\r{\rho}
\def\s{\sigma}
\def\t{\tau}
\def\x{\xi}
\def\w{\omega}
\def\W{\Omega}
% Nice greek
\newcommand{\p}{\varphi}
\newcommand{\e}{\varepsilon}

% Shorcut vectors
\def\vx{\vect{x}}
\def\vy{\vect{y}}
\def\va{\vect{a}}
\def\vb{\vect{b}}

\newcommand{\CC}{\mathbb C}
\newcommand{\FF}{\mathbb F}
\newcommand{\RR}{\mathbb R}
\newcommand{\NN}{\mathbb N}
\newcommand{\PP}{\mathbbm P}
\newcommand{\EE}{\mathbbm E}
\newcommand{\TT}{\mathbbm T}
\newcommand{\VV}{\mathbbm V}
\newcommand{\QQ}{\mathbb Q}
\newcommand{\WW}{\mathbbm W}
\newcommand{\ZZ}{\mathbbm Z}
\renewcommand{\SS}{\mathbbm S}

% Expectation/Probability
\newcommand{\ee}[1]{\mathbbm{E}[{#1}]}
\newcommand{\pp}[1]{\mathbbm{P}({#1})}

\newcommand{\GG}{\mathsf G}
\newcommand{\XX}{\mathsf X}
\renewcommand{\AA}{\mathsf A}
\newcommand{\YY}{\mathsf Y}
\newcommand{\ZZZ}{\mathsf Z}

\newcommand{\aA}{\mathscr A}
\newcommand{\iI}{\mathscr I}
\newcommand{\eE}{\mathscr E}
\newcommand{\fF}{\mathscr F}
\newcommand{\rR}{\mathscr R}
\newcommand{\lL}{\mathscr L}
\newcommand{\cG}{\mathscr G}

\newcommand{\pP}{\mathcal P}
\newcommand{\aAA}{\mathcal A}
\newcommand{\vV}{\mathcal V}
\newcommand{\mM}{\mathcal M}
\newcommand{\oO}{\mathcal O}
\newcommand{\gG}{\mathcal G}
\newcommand{\hH}{\mathcal H}
\newcommand{\tT}{\mathcal T}
\newcommand{\bB}{\mathcal B}
\newcommand{\zZ}{\mathcal Z}
\newcommand{\cC}{\mathcal C}
\newcommand{\dD}{\mathcal D}
\newcommand{\wW}{\mathcal W}
\newcommand{\uU}{\mathcal U}
\newcommand{\sS}{\mathcal S}

% Common collections
\def\cA{\col{A}}
\def\cB{\col{B}}
% \def\cC{\col{C}}
\def\cT{\col{T}}
\def\cU{\col{U}}

% Common closures
\def\clA{\closure{A}}
\def\clB{\closure{B}}
\def\clK{\closure{K}}

% operators
\DeclareMathOperator{\cl}{cl}
\DeclareMathOperator{\graph}{graph}
\DeclareMathOperator{\interior}{int}
\DeclareMathOperator{\Prob}{Prob}
\DeclareMathOperator{\determinant}{det}
\DeclareMathOperator{\trace}{trace}
\DeclareMathOperator{\sgn}{sgn}
\DeclareMathOperator{\Span}{span}
\DeclareMathOperator{\diag}{diag}
\DeclareMathOperator{\proj}{proj}
\DeclareMathOperator{\rank}{rank}
\DeclareMathOperator{\cov}{Cov}
\DeclareMathOperator{\corr}{Corr}
\DeclareMathOperator{\var}{Var}
\DeclareMathOperator{\mse}{mse}
\DeclareMathOperator{\se}{se}
\DeclareMathOperator{\row}{row}
\DeclareMathOperator{\col}{col}
\DeclareMathOperator{\range}{rng}
\DeclareMathOperator{\kernel}{ker}
\DeclareMathOperator{\dimension}{dim}
\DeclareMathOperator{\bias}{bias}
\DeclareMathOperator{\dom}{dom}
\DeclareMathOperator{\ran}{ran}
\DeclareMathOperator{\Int}{Int}
\DeclareMathOperator{\Cl}{Cl}
\DeclareMathOperator{\im}{im}
\DeclareMathOperator{\conv}{conv}

% \end{extract*}

\title{Measure Theory - Problem set 1 - Week 3 }
\author{Thomas Pellet}
\begin{document}
\maketitle



\section{Measure Spaces}

\subsection*{Exercise 1.3}

$\mathcal{G}_{1}=\{A : A \subset \mathbb{R}, A \text { open }\}$ is the set of open sets. It therefore includes the empty set and its complement $\RR$ which are both open. The complements of open sets are open. $\mathcal{G}_{1}$ is therefore closed under complement. Union of open sets are also open. This set is therefore an algebra. Countable unions of open sets are also open. $\mathcal{G}_{1}$ is therefore a $\s$-algebra. \\
$\mathcal{G}_{2}=\{A : A \text { is a finite union of intervals of the form }(\mathrm{a}, \mathrm{b}],(-\infty, \mathrm{b}], \text { and }(\mathrm{a}, \infty)\}$ is an algebra because it includes the empty set for $a=b$, includes complements of intervals and by construction includes finite unions of intervals. It is not a $\s$-algebra because it does not include countable unions. \\

$\mathcal{G}_{2}=\{A : A \text { is a countable union of intervals of the form }(\mathrm{a}, \mathrm{b}],(-\infty, \mathrm{b}], \text { and }(\mathrm{a}, \infty)\}$ is an algebra because it includes the empty set for $a=b$, includes complements of intervals and by construction includes finite unions of intervals. It is a $\s$-algebra because it does include countable unions.

\subsection*{Exercise 1.7}
Suppose $\exists B$ a $\sigma$-algebra in X such that $B \nss \pP(X)$. Then $\exists$ $A \in B \st A \notin \pP(X)$. \\
$\therefore A \notin X$, contradiction. \\

Suppose $\exists / S$ a $\sigma$-algebra in X such that $\{\emptyset ,X\}$ $\nss S$. If the empty set is not in S we have a contradiction. If $X\notin S$ Then $\emptyset^c \notin S$ and we have another contradiction.

\subsection*{Exercise 1.10}

Let $\braces{S_\a}$ be a family of $\s$-algebras on X. The empty set is included in all $S_\a$ and is therefore part of the intersection. 
Let $A \in \cap S_\a$, A is in all the elements of the family $\braces{S_\a}$. It is therefore closed under countable union and complement in each $S_\a$, which are therefore part of their intersection. 

\subsection*{Exercise 1.22}

\begin{itemize}
	\item monotonicity $ \mu (B) = \mu (A \cup A^c\cap B) = \mu (A) + \mu (A^c\cap B) \geq \mu (A) $
	\item subadditivity: For n = 2

\end{itemize}
\begin{align}
\mu (A_1) + \mu (A_2) &= \mu(A_1 \setminus A_2) + \mu(A_2 \setminus A_1) + 2 \mu(A_1 \cap A_2) \\
\mu \parens{\cup A_i} &= \mu(A_1 \setminus A_2) + \mu(A_2 \setminus A_1) + \mu(A_1 \cap A_2) \\
\imp \mu \parens{\cup^2_1 A_i} &\leq \mu (A_1) + \mu (A_2)
\end{align}
By iteration, one can prove that it is true for a countable union of sets.

\subsection*{Exercise 1.23}

Let $A,B \ \in \ S $ such that $\l (A) = \mu(A\cap B)$
\begin{itemize}
	\item $\l (\emptyset) = \mu( \emptyset \cap B) = 0 $
	\item Let $\braces{A_i}$ be a family of disjoint sets in S: 
	\begin{align*}
	\l (\cup^\infty_1 A_i) &= \mu( \parens{\cup^\infty_1 A_i} \cap B) \\
						   &= \mu(\cup^\infty_1 A_i \cap B)\\
						   &= \sum^\infty_1 \mu(A_i \cap B) \\
						   &= \sum^\infty_1 \l(A_i)
	\end{align*}
	
\end{itemize}

\subsection*{Exercise 1.26}
	\begin{align*}
	\mu (\cap^\infty_{i=1} A_i) &= \mu \parens{\parens{\cup^\infty_1 A^c_i}^c} \\
						   &=  \mu (X) - \mu \parens{\cup^\infty_1 A^c_i}\\
						   &= \mu (X) - \lim_{n \to \infty} \mu \parens{A^c_n} \quad \text{using} \ i) \\
						   &= \mu (X) - \mu (X) +\lim_{n \to \infty} \mu \parens{A_n} \\
						   &=  \lim_{n \to \infty} \mu \parens{A_n}
	\end{align*}


\section*{Lebesgue Measure}
\subsection*{Exercise 2.10}
\begin{align}
\mu \parens{B} = \mu \parens{B \cap E \cup B \cap E^c} \leq \mu \parens{B \cap E} + \parens{B \cap E^c} \quad \text{by subadditivity. Equality holds}
\end{align}

\subsection*{Exercise 2.14}

Using Cathéodory extension, we need to prove that $\bB (\RR) = \s (\AA)$

By definition $ \mathcal{A}=\{A : A \text { is a finite disjoint union of intervals } (\mathrm{a}, \mathrm{b}],(-\infty, \mathrm{b}], \text { and }(\mathrm{a}, \infty) \}$. The $\s-$algebra generated by this set must include all open and closed set in R to be closed under complement. Therefore we have that $\bB (\RR) \sse \s (\AA)$. We need to prove the second inclusion. \\
Let $B \in \bB (\RR)$. B is covered by an open set so $B \sse (a,b)$ so $B \in \s(\mathcal{A})$

\section*{Measurable functions}

\subsection*{Exercise 3.1}
Let $\braces{x} \ss \RR$  be a singleton set. It is therefore a countable set. Suppose that $\exists \ \e$ such that $\mu (x) > \e $ with $\mu$ the Lebesgue measure. 
By construction:
$\braces{x} \ss \squares{x - \frac{\e}{2}, x + \frac{\e}{2}}$. Using subadditivity we have that :

\begin{align}
\mu (x) &\leq \mu \parens{\squares{x - \frac{\e}{2}, x + \frac{\e}{2}}}
		& \leq \e
\end{align}

This is a contradiction. Therefore, the measure of a singleton set is zero. By definition, countable sets are the disjoint union of singleton sets. The countable sum of zero measures is zero and the countable union of countable sets is therefore zero by iteration.  

\subsection*{Exercise 3.7}
Suppose $f: X \to \RR$ is measurable and define $g(x) = - f(x)$.

For any a in $\RR$ We have that $\braces{x \in X : f(x) <a}$ is measurable. This is equivalent to  $\braces{x \in X : -g(x) <a}= \braces{x \in X : -g(x) >a}$ being measurable. \\
Statements with non-strict equalities hold because $\mM$ is a $\s$-algebra so that $f^{-1} ((-\infty,a)) = f^{-1} ([a,\infty))$ is also in $\mM$ and therefore also measurable. Hence, $\mM$ contains closed sets and for any $a \in \braces{x \in X : -g(x) \leq a}$ is measurable. \\

The $\s-$algebra we are implicitly using here is the Borel algebra over $\RR$

\subsection*{Exercise 3.10}

Supposing that f and g are measurable functions, for every a, we have that: 

\begin{align}
\braces{ x \in X: f(x) + g(x) < a} &= \cup_{r \in \QQ} \parens{ \braces{x \in X: g(x) < a - r} \cap \braces{ x \in X: f(x) < r} }
		& \leq \e
\end{align}

This is the countable union of intersecting measurable sets, it is therefore measurable as well. 

Similarly: 

\begin{align}
\braces{ x \in X: f(x) g(x) < a} &= \braces{x \in X: \frac{1}{4} \parens{(f(x) + g(x))^2 - (f(x) - g(x))^2 < a}} 
\end{align}

and therefore the product of measurable functions is measurable given that the square of measurable functions is measurable. 
\newpage

\subsection*{Exercise 3.17}

Suppose that f is bounded. $\exists c \ st,\forall x \in X \quad f(x) \leq c$
Given that f is bounded by c.
\begin{align}
\abs{s_c(x) - f(x)} &= \abs{\sum_{i=1}^{c \cdot 2^{c}} \frac{i-1}{2^{c}} \chi_{E_{i}^{c}}+ c \chi_{E_{\infty}^{c}} - f(x)} \\
					&= \abs{\sum_{i=1}^{c \cdot 2^{c}} \frac{i-1}{2^{c}} \chi_{E_{i}^{c}} - f(x)_{|f(x)<c}}
\end{align}


Note that for any x, since $f(x)$ is bounded $f(x) \in [0,c]$ so $x \in E_{i}^{c}$ for some i. Note there is an $N \geq c$ such that $\frac{1}{2^{N}}< \epsilon$. \\
 Then for any $n \geq N$, $\left|f(x)-s_{n}(x)\right|<\epsilon$, so we have uniforme convergence.

 \section*{Lebesgue Convergence}

\subsection*{Exercise 4.13}
Because $|f|<M \text { on } E \in \mathcal{M}$ and $\mu(E)<\infty$ we have using 4.7:
\begin{align}
\int_{E} f^+ d \mu &< \int_{E} M \chi_{E_{M}} d \mu = M \mu (E_M) < \infty \\
\int_{E} f^- d \mu &< \int_{E} M \chi_{E_{M}} d \mu = M \mu (E_M) < \infty
\end{align}

F is therefore integrable and $f \in \mathscr{L}^{1}(\mu, E)$.

\subsection*{Exercise 4.14}

Suppose $f \in \mathscr{L}^{1}(\mu, E)$. \\
Let $ E_n = \braces{x \in X: f(x) \geq n}$ and $E = \cup^{\infty}_{n=1} E_n$ Using the properties of the integral, we have that:

\begin{align}
\int_{X} f d \mu &\geq \int_{E_n} f d \mu \geq \int_{E_n} n \chi_{E_{n}} d \mu = n \mu(E_n) \geq n \mu(E) \\
\imp \mu(E) &\leq \int_{X} \frac{1}{n} f d \mu \sim \oO (\frac{1}{n})
\end{align}
This tends to zero as $n \to \infty$



\subsection*{Exercise 4.15}



Suppose $f,g \in \mathscr{L}^{1}(\mu, E)$ and $f \leq g$: 
\begin{align}
\braces{\int_{E} s d \mu: 0 \leq s \leq f, \text{s simple}} &\ss \braces{\int_{E} s d \mu: 0 \leq s \leq g, \text{s simple}}
\end{align}
Taking the supremum and by definition of the integral, we have that

\begin{align}
\int_{E} f d \mu \leq \int_{E} g d\mu
\end{align}

\subsection*{Exercise 4.16}
Suppose $f \in \mathscr{L}^{1}(\mu, E)$ and $A \ss E$
we have that: 


\begin{align}
\int_{A} f^+ d \mu &\leq \int_{E} f^+ d d \mu < \infty \\
\int_{A} f^- d \mu &leq \int_{E} f^- d \mu  < \infty
\end{align}


and therefore $f \in \mathscr{L}^{1}(\mu, A)$

\subsection*{Exercise 4.21}
If $A, B \in \mathcal{M}, B \subset A \text { and } \mu(A-B)=0, \text { then if } f \in \mathscr{L}^{1}$:

\begin{align}
\int_{A} f^+ d \mu - \int_{B} f^+ d \mu &= \int_{A\cap B \cup A \cap B^C} f^+ d \mu - \int_{B} f^+ d  \\
 		&= \int_{B} f^+ d \int_{A \cap B^C} f^+ d \mu - \int_{B} f^+ d \\
 		& = \int_{A \cap B^C} f^+ d \mu \leq c \mu (A \cap B^C) = 0
\end{align}
The two integrals are therefore identical. The same is true for $f^-$.




\end{document}